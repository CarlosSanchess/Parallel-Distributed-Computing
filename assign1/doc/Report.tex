\documentclass{article}
\usepackage{graphicx} % Required for inserting images

\title{Report for the First Project of Computer Parallel and Distributed Computation } 

\author{
Turma 11 Grupo \_\_ \\
João Rebelo, up202107209 \\
Carlos Sanches, up202107694 \\
Luís Ganço, up202004196
}
\date{March 03, 2025}

\begin{document}

\newpage 

\maketitle
\begin{center}
    \large \textit{Project Theme:} Performance Evaluation of Matrix Multiplication Algorithm
\end{center}
\newpage
% Add the table of contents (index)
\tableofcontents
\newpage % Start a new page after the table of contents

\section{Introduction}

\subsection{Problem Description}
In modern computing, the performance of processors is heavily influenced by the memory hierarchy, particularly when handling large datasets. Matrix multiplication, a fundamental operation in many scientific and engineering applications, serves as case study to explore these performance dynamics. This project aims to evaluate the impact of memory hierarchy on processor performance by implementing and analyzing different versions of matrix multiplication algorithms in multiple programming languages.

The project is divided into two main parts: single-core and multi-core performance evaluations. In the single-core evaluation, we implement and compare matrix multiplication algorithms in C/C++ with Java. We measure processing times for matrices of varying sizes and analyze the effects of different algorithmic approaches, including a block-oriented implementation. The Performance API (PAPI) is utilized to collect relevant performance metrics, such as cache misses and floating-point operations, to provide deeper insights into the behavior of the memory hierarchy.

For the multi-core evaluation, we parallelize the matrix multiplication algorithms using OpenMP and analyze their performance in terms of MFlops, speedup, and efficiency. Two parallelization strategies are explored, and their results are compared to understand the trade-offs between different parallel implementations.

This report documents the implementation details, performance metrics, and analysis of the results obtained from both single-core and multi-core evaluations. 

\subsection{Algorithms explanation}


\section{Performance Metrics}

% Add content for the Performance Metrics section here.

\section{Results and Analysis}

% Add content for the Results and Analysis section here.

\section{Conclusions}

% Add content for the Conclusions section here.

\end{document}